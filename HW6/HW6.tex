\documentclass{article}

\usepackage[english]{babel}
\usepackage[utf8]{inputenc}
\usepackage{amsmath,amssymb}
\usepackage{parskip}
\usepackage{graphicx}
\usepackage{hyperref}
% Margins
\usepackage[top=2.5cm, left=3cm, right=3cm, bottom=4.0cm]{geometry}
% Colour table cells
\usepackage[table]{xcolor}

% Get larger line spacing in table
\newcommand{\tablespace}{\\[1.25mm]}
\newcommand\Tstrut{\rule{0pt}{2.6ex}}         % = `top' strut
\newcommand\tstrut{\rule{0pt}{2.0ex}}         % = `top' strut
\newcommand\Bstrut{\rule[-0.9ex]{0pt}{0pt}}   % = `bottom' strut

%%%%%%%%%%%%%%%%%
%     Title     %
%%%%%%%%%%%%%%%%%
\title{Exercise 6 \\ Probability Theory 2020 Autumn}
\author{Hanmo Chen \\ 2020214276}
\date{\today}

\begin{document}
\maketitle

\section{Problem 1}

\begin{equation}
    P(S_i\geqslant 0, \forall 1\leqslant i \leqslant 2n |S_{2n} = 0 ) = \frac{P(S_i\geqslant 0, \forall 1\leqslant i \leqslant 2n-1 ,S_{2n} = 0 )}{P(S_{2n} = 0)}
\end{equation}

Consider the event $\{S_i\geqslant 0, \forall 1\leqslant i \leqslant 2n-1 ,S_{2n} = 0\}$ partitioned by its first return to zero,

\begin{equation}
    \left\{S_i\geqslant 0, \forall 1\leqslant i \leqslant 2n-1 ,S_{2n} = 0\right\} = \bigcup_{j=1}^{n}\left\{S_i> 0, \forall 1\leqslant i \leqslant 2j-1 ,S_{2j} = 0,S_{i} \geqslant 0,\forall 2j+1\leqslant i \leqslant 2n-1,S_{2n} = 0 \right\}
\end{equation}

Denote $p_{2n} = P(S_i\geqslant 0, \forall 1\leqslant i \leqslant 2n-1 ,S_{2n} = 0 )$ and $q_{2n} = P(S_i> 0, \forall 1\leqslant i \leqslant 2n-1 ,S_{2n} = 0 )  = \frac{1}{2} P(S_i\neq 0, \forall 1\leqslant i \leqslant 2n-1 ,S_{2n} = 0 ) = \frac{1}{2} f_{2n}$, 



\begin{equation}
    p_{2n} = \sum_{i=1}^n p_{2n-2i} q_{2i}
\end{equation}

Define $P(z)= \sum_{n=0}^{\infty} p_{2n} z^n , Q(z) = \sum_{n=1}^{\infty}q_{2n} z^n = \frac{1}{2} F(z)$


\begin{equation}
    P(z) = \frac{1}{1- Q (z)} = \frac{2}{1+\sqrt{1-z}} = \frac{2}{z} (1-\sqrt{1-z}) =\frac{2}{z} F(z)
\end{equation}


So $p_{2n} =2f_{2n+2} = \frac{2}{2n+1} \binom{2n+2}{n+1} \frac{1}{2^{2n+2}}$. Thus,

\begin{equation}
    P(S_i\geqslant 0, \forall 1\leqslant i \leqslant 2n |S_{2n} = 0 ) =  \frac{p_{2n}}{u_{2n}} = \frac{1}{n+1}
\end{equation}


\section{Problem 2}
$L_{\max}^{(n)}$ is the longest length of success runs in $X_1,X_2,\cdots,X_n$.\footnote{Reference: Erdös, P., \& Rényi, A. (1970). On a new law of large numbers. Journal d’analyse mathématique, 23(1), 103-111.}

Denote $S_0 = 0,S_n = \sum\limits_{i=1}^n X_i $ and define 

\begin{equation}
    \Theta(n,k)  = \max_{0\leqslant i \leqslant n-k} \frac{S_{i+k}-S_{i}}{k}
\end{equation}

So 

\begin{equation}
    L_{\max}^{(n)} \geqslant k \Longleftrightarrow \Theta(n,k) \geqslant 1
\end{equation}

% Take $k$ as $[c\log n ]$, so $k \to \infty$ as $n \to \infty$, using Sanov's Theorem\footnote{Reference: \url{https://en.wikipedia.org/wiki/Sanov\%27s_theorem}} in Information Theory, or using the results from Bahadur and Rao\footnote{Bahadur, R. R., \& Rao, R. R. (1960). On deviations of the sample mean. Ann. Math. Statist, 31(4), 1015-1027.}, we have

% \begin{equation}
%     \lim_{k\to \infty} - \frac{1}{k} \log(P(\frac{S_{i+k}-S_{i}}{k} \geqslant 1)) = D_{KL}(\text{Bernoulli}(1) ||\text{Bernoulli}(p) ) = -\log p 
% \end{equation}

% Thus,

\begin{equation}
    P(\frac{S_{i+k}-S_{i}}{k}\geqslant 1) = p^k = e^{k\log p }
    % \sim \exp\left(k\log p + o(1)\right),  k \to \infty
\end{equation}

And 
\begin{equation}
    \begin{aligned}
        P( L_{\max}^{(n)} \geqslant k) & = P(\Theta(n,k) \geqslant 1) = P\left(\max_{0\leqslant i \leqslant n-k} \frac{S_{i+k}-S_{i}}{k} \geqslant 1 \right)  \\
        &= 1- P\left(\max_{0\leqslant i \leqslant n-k} \frac{S_{i+k}-S_{i}}{k} < 1 \right) = 1- P\left(\frac{S_{i+k}-S_{i}}{k} < 1 , 0\leqslant i \leqslant n-k \right)
    \end{aligned}
\end{equation}

Now we are trying to find the upper and lower bound of $P\left(\frac{S_{i+k}-S_{i}}{k} < 1 , 0\leqslant i \leqslant n-k \right)$. First,

\begin{equation}
    \begin{aligned}
        P\left(\frac{S_{i+k}-S_{i}}{k} < 1 , 0\leqslant i \leqslant n-k \right) & \leqslant P\left(\frac{S_{(i+1)k}-S_{ik}}{k} < 1 ,  i=0,1 \cdots [\frac{n-k}{k}] \right) \\
        & \leqslant (1-e^{k\log p})^{n/k} = (1-n^{c\log p})^{n/k} \\
        & \leqslant \exp\left({-\frac{n^{1-c\log \frac{1}{p}}}{c\log n }}\right)
    \end{aligned}
\end{equation}

Then denote $A_i$ as the event $\left\{\frac{S_{i+k}-S_{i}}{k} < 1\right\}$
 
\begin{equation}
    \begin{aligned}
        P\left(\frac{S_{i+k}-S_{i}}{k} < 1 , 0\leqslant i \leqslant n-k \right) & = P(\bigcap_{i=0}^{n-k} A_i) = 1- P(\bigcup_{i=0}^{n-k} \overline{A_i}) \\
        & \geqslant 1 - \sum_{i=1}^n P( \overline{A_i}) = 1-np^k = 1- n^{1-c\log \frac{1}{p}}
    \end{aligned}
\end{equation}

If $c<\frac{1}{\log \frac{1}{p}}$, 

\begin{equation}
    \lim_{n\to \infty}  P\left(\frac{S_{i+k}-S_{i}}{k} < 1 , 0\leqslant i \leqslant n-k \right) \leqslant  \lim_{n\to\infty} \exp\left({-\frac{n^{1-c\log \frac{1}{p}}}{c\log n }}\right) = 0
\end{equation}

Thus $\forall c <\frac{1}{\log \frac{1}{p}},\lim\limits_{n\to \infty} P(L_{\max}^{(n)} \geqslant  c\log n ) = 1 $

If $c> \frac{1}{\log \frac{1}{p}}$,

\begin{equation}
    \lim_{n\to \infty}  P\left(\frac{S_{i+k}-S_{i}}{k} < 1 , 0\leqslant i \leqslant n-k \right) \geqslant \lim_{n\to \infty}  1- n^{1-c\log \frac{1}{p}} = 1
\end{equation}

Thus $\forall c >\frac{1}{\log \frac{1}{p}},\lim\limits_{n\to \infty} P(L_{\max}^{(n)} \geqslant  c\log n ) = 0 $. $f(p) = \frac{1}{\log \frac{1}{p}}$.



\end{document} 
