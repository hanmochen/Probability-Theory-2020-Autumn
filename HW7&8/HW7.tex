\documentclass{article}

\usepackage[english]{babel}
\usepackage[utf8]{inputenc}
\usepackage{amsmath,amssymb}
\usepackage{parskip}
\usepackage{graphicx}
\usepackage{hyperref}
% \usepackage{unicode-math}
\usepackage{dsfont}
% \usepackage{bbm}
% Margins
\usepackage[top=2.5cm, left=3cm, right=3cm, bottom=4.0cm]{geometry}
% Colour table cells
\usepackage[table]{xcolor}

% Get larger line spacing in table
\newcommand{\tablespace}{\\[1.25mm]}
\newcommand\Tstrut{\rule{0pt}{2.6ex}}         % = `top' strut
\newcommand\tstrut{\rule{0pt}{2.0ex}}         % = `top' strut
\newcommand\Bstrut{\rule[-0.9ex]{0pt}{0pt}}   % = `bottom' strut
\DeclareMathOperator{\E}{\mathbb{E}}
\DeclareMathOperator{\Var}{\operatorname{Var}}
\DeclareMathOperator{\Cov}{\operatorname{Cov}}
\newcommand{\vu}{\boldsymbol{u}}
%%%%%%%%%%%%%%%%%
%     Title     %
%%%%%%%%%%%%%%%%%
\title{Exercise 7 \& 8 \\ Probability Theory 2020 Autumn}
\author{Hanmo Chen \\ 2020214276}
\date{\today}

\begin{document}
\maketitle

\section{Problem 1}

Denote the entries in $U$ as $u_{ij}$ and entries in $Y$ as $Y_{ij}$,so 

\begin{equation}
    Y_{ij} = \sum_{r,s} u_{ri} u_{sj} X_{ij} 
\end{equation}

Also we have,

\begin{equation}
    \Cov(X_{ij},X_{mn}) = \left\{\begin{aligned}
        &2,\quad i=j=m=n \\
        &1,\quad (i,j) = (m,n) \text{ or } (i,j) = (n,m), i \neq j \\
        &0,\quad \text{otherwise}
    \end{aligned}\right.
\end{equation}

Thus

\begin{equation}
    \begin{aligned}
        \Cov(Y_{ij},Y_{mn}) & = \Cov\left(\sum_{r,s} u_{ri} u_{sj} X_{ij}, \sum_{p,q} u_{pm} u_{qn} X_{mn} \right) \\
        & = 2\sum_{r=1}^n u_{ri}u_{rj} u_{rm}u_{rn} + \sum_{r\neq s} u_{ri} u_{sj} u_{rm} u_{sn} + \sum_{r\neq s} u_{ri} u_{sj} u_{rn} u_{sm} \\
        & = \sum_{r,s}\left[u_{ri} u_{sj} u_{rm} u_{sn}+u_{ri} u_{sj} u_{rn} u_{sm} \right] \\
        & = \left(\sum_{r} u_{ri} u_{rm} \right) \left(\sum_{s} u_{sj} u_{rn} \right) + \left(\sum_{r} u_{ri} u_{rn} \right) \left(\sum_{s} u_{sj} u_{sm} \right) 
    \end{aligned}
\end{equation}

Denote the column vectors in $U$ as $\vu_i,i=1,2,3,\cdots,N$, so $\vu_i\cdot \vu_j = \sum_{r} u_{ri} u_{rj} = \delta_{ij}$. 

\begin{equation}
    \Cov(Y_{ij},Y_{mn}) = \delta_{im} \delta_{jn} +  \delta_{jm} \delta_{in} = \left\{ \begin{aligned}
        &2,\quad i=j=m=n \\
        &1,\quad (i,j) = (m,n) \text{ or } (i,j) = (n,m), i \neq j \\
        &0,\quad \text{otherwise}
    \end{aligned}\right.
\end{equation}

Because $X_{ij}, j \geqslant i $ are independent Gaussian variables, so the joint distribution of $Y_{ij}$ is joint Gaussian distribution, which means,

\begin{equation}
    \Cov(Y_{ij},Y_{mn}) = 0 \Longleftrightarrow Y_{ij},Y_{mn} \text{ are independent}
\end{equation}

So all the entries on and above the diagonal of $Y$ are independent, and $Y_{ii} \sim N(0,2), i =1,2,3,\cdots,N$ and $Y_{ij} \sim N(0,1), 1\leqslant i < j \leqslant n$. (It is easy to see that $\E[Y_{ij}] = 0$)

\section{Problem 2}


\end{document} 
